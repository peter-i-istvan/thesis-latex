%----------------------------------------------------------------------------
\chapter*{\bevezetes}
%----------------------------------------------------------------------------

% A bevezető tartalmazza a diplomaterv-kiírás elemzését, történelmi előzményeit, a feladat indokoltságát (a motiváció leírását), az eddigi megoldásokat, és ennek tükrében a hallgató megoldásának összefoglalását.

% A bevezető szokás szerint a diplomaterv felépítésével záródik, azaz annak rövid leírásával, hogy melyik fejezet mivel foglalkozik.

\section*{Computer vision and its subtasks}

Computer Vision is a branch of Artificial Intelligence aimed at deducing scene understanding and other higher-order information from visual input like images, videos, or more specialised sensor data like LIDAR point clouds etc. Possible subtasks in Computer Vision are classification of images, detection of objects in images and video, segmentation, pose estimation of specific entities, tracking etc. This thesis focuses on two specific tasks: object detection and multiple object tracking. Both are widely discussed topics, having decades of research behind them.

In short, object detection has the aim to find all relevant objects on images, usually indicating their location as bounding boxes that fit the contours of the object. Since the mid 2010s, deep neural networks saw an exponential rise in popularity and performance, by fixing numerical problems that previously plagued these kinds of machine learning models, and having access to a larger amount of data and more computing power than ever before. Particularly in vision applications, convolutional filters used to find certain features could be adapted as neural network layers, and the filters themselves became trainable parameters. This new approach gave rise to the VGG and AlexNet architectures, among others, kicking off the era of Convolutional Neural Networks (CNN). Today, almost all intelligent computer vision applications use some sorts of CNNs, so it is natural that the most popular and performant object detectors are also built on CNNs. Previously, systems using hand-crafted features like the Deformable Parts Model (DPM) were widely used.

Multiple object tracking works on videos, and aims to determine whole trajectories of objects of interest. The main challenge for multiple object trackers besides detection is preserving object identity, guaranteeing that the same object was tracked for the whole duration of the trajectory.

\section*{Goal and motivation}

This thesis was made with the goal of comparing two paradigms in modern object detection architectures: the older, more estabilished, Fully Convolutional Network-based (FCN) detectors versus the relatively recent Transformer-based implementations.

The comparison itself, considering only the classic detection metrics used in the field, is an active topic in the research community that spawned a myriad of analises, varying in depth and scope. My goal wasn't to make yet another analysis of the kind, I wanted rather to measure performance in another computer vision task where results of the object detection are merely inputs to the solution, but the quality of the detection can make or break the performance of the whole system. This task of my choice was multiple object tracking (MOT).

In the Project Laboratory course of the previous semester, I implemented a version of a well-known multiple object tracker called the Simple Online Realtime Tracker (SORT) for a specific task involving multiple fisheye cameras placed in a parking space. In that context, the object detector, that was based on the You Only Look Once (YOLO) architecture, was considered as a given, provided to me by the company that hosted my topic.

Starting from that setup, and following the suggestion of my advisor at the company, I considered to extend the scope of my examination of the field by comparing multiple possible object detectors, with special emphasis on a new emerging approach: using the Tranformer architecture in vision tasks.

For the new project, I have chosen to work on public MOT databases, but staying in the topic of road traffic analysis. I worked with the UA-DETRAC database, that consists of videos taken from fixed camera positions over highways and highly frequented roads. For the tracker, I kept using SORT due to it being a simple yet highly performant solution. Moving away from the custom implementation, I chose to use the official implementation written by the authors of the SORT research paper. I evaluated the tracker with seven detectors, two Detection Transformer (DETR) and five YOLOv5 variants, the main difference between models of the same type being model size, as in number of parameters, which influences inference time as well. The benchmarking system for the UA-DETRAC database is currently not available, so I had to write my own implementation of the benchmark, based on the metrics proposed by the paper that introduced the UA-DETRAC benchmark itself. In this thesis, I will present the final results of this assessment.

As for the theoretical parts, I gave a brief overview of the model architectures, summarizing the research papers that led to the development of the two most popular braches of object detectors. I analyze available implementations of particular models, exploring the available possibilities to train on custom data, called trasfer learning. I also review popular training datasets, specially the Microsoft Common Objects in Context (MS COCO or simply COCO) dataset and benchmark, considered the most popular, large-scale and richly annotated training database for object detection, classification and semantic segmentation.

\section*{Thesis structure}

The first chapter of the thesis contains a more detailed introduction to the problem of object detection, focusing on standard metrics for assessment as well. After that, I review the modern fully convolutional object detectors, differentiating between two-stage and one-stage detectors, but only going into details in the case of the latter. Among the one-stage detectors, I chose the YOLO architecture for exploration, following its main versions, ideas, and overall architecture. Moving away from FCNs, I summarize the history of the Transformer, first introduced in natural language processing, its adaptation to computer vision and finally the DETR, the first end-to-end object detector, where the separate non-maximum supression step from FCNs is replaced by a novel training loss.

The second chapter is the overview of available implementations in the most popular machine learning frameworks, as well as training databases for object detection, and the possibility of training the implementations on custom data, a common use case in the industry.

The third, largest chapter contains a theoretical overview of the MOT problem, citing a recent literature review. Then I explore the chosen benchmark and dataset, the UA-DETRAC, detailing the metrics used for evaluation. Then, I summarize the theoretical background of the SORT architecture, followed by the description of my own work: the measurement, its implementation and results.