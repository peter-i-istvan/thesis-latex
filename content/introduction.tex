%----------------------------------------------------------------------------
\chapter{\bevezetes}
%----------------------------------------------------------------------------

A bevezető tartalmazza a diplomaterv-kiírás elemzését, történelmi előzményeit, a feladat indokoltságát (a motiváció leírását), az eddigi megoldásokat, és ennek tükrében a hallgató megoldásának összefoglalását.

A bevezető szokás szerint a diplomaterv felépítésével záródik, azaz annak rövid leírásával, hogy melyik fejezet mivel foglalkozik.

\section{The field of Computer Vision}

Computer Vision is a branch of Artificial Intelligence aimed at deducting higher-order information from visual input like images, videos, or more specialised sensor data like LIDAR point clouds etc. Some individual tasks in Computer Vision are image classification, object detection, segmentation, pose estimation of specific entities etc.

\section{Object Detection}
Blah Blah

\section{My Goal}

My goal in this thesis is to give an overview of the differences between already estabilished Fully Convolutional Neural Networks (FCN) compared to upcoming Transformer-based architectures in the task of Object Detection. In the former, i restrict myself to single-stage detectors, namely the YOLO architecture.